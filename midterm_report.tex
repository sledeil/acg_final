\documentclass[sigconf]{acmart}

\usepackage{graphicx}
\usepackage{float} % for [H] in figures
\usepackage{hyperref}
\usepackage{caption}
\usepackage{subcaption}
\usepackage{tabularx} 
\usepackage{enumitem} % add this to preamble

% 设置 itemize 的行间距紧凑
\setlist[itemize]{topsep=2pt, partopsep=0pt, parsep=0pt, itemsep=2pt}
% \addtolength{\parskip}{-0.3ex}
\setlength{\parskip}{0pt}
\setlength{\itemsep}{0pt}

%% Metadata
\title{ACG Project Midterm Report: Interactive Space Navigation Game}

\author{Sihan Wang}
\authornote{Both authors contributed equally to this work.}
\email{sh-wang23@mails.tsinghua.edu.cn}
\affiliation{%
  \institution{2023011415, Yao Class 32, IIIS}
  \city{Beijing}
  \country{China}
}

\author{Honghua Chen}
\authornotemark[1]
\email{chh25@mails.tsinghua.edu.cn}
\affiliation{%
  \institution{2025040146, Yao Class 50, IIIS}
  \city{Beijing}
  \country{China}
}

%% Remove ACM copyright (for course projects)
\setcopyright{none}
\acmConference[]{ACG Course Project}{2025}{}
\acmYear{2025}
\acmDOI{}
\acmISBN{}

\begin{document}

\begin{abstract}
This report presents our Interactive Game project: a physics-based space navigation game that simulates realistic gravitational mechanics, including orbital maneuvers, gravity assists, and trajectory planning. The game is implemented entirely from scratch using Three.js and a custom physics integrator.
\end{abstract}

\maketitle

\section{Game Overview}

We introduce an \textbf{interactive, physics-based space navigation game} that highlights realistic gravitational dynamics. Unlike traditional arcade-style space games, our design focuses on orbital mechanics, gravitational slingshots, and long-term trajectory planning. This allows players to intuitively experience astrophysical phenomena such as Hohmann transfers, gravity assists, perturbed orbits, multi-body influences, and comet interception trajectories.

The player controls a spacecraft navigating within a simulated solar system. By applying thrust, the player can adjust the spacecraft’s instantaneous velocity, while Newtonian gravity from surrounding celestial bodies continuously shapes its path. The interaction between thrust and gravitational forces produces orbital motion that closely resembles real-world physics. Players must strategically leverage gravity to minimize fuel consumption.

To guide players through increasing complexity, we design a set of progressively challenging missions:

\begin{itemize}
    \item \textbf{Earth--Moon Hohmann Transfer:} A beginner task that teaches basic maneuver control \cite{farquhar1980trajectories}.
    \item \textbf{Lunar Gravity Assist $\rightarrow$ Mars:} Inspired by \textit{The Martian}, this task demonstrates how slingshot mechanics enable long-distance travel \cite{farquhar1984aiaa}.
    \item \textbf{Halley’s Comet Rendezvous Mission:} A tribute to past missions by USSR, USA, Japan, and ESA using different trajectory strategies \cite{dunham2014isee}.
\end{itemize}

Beyond Newtonian gameplay, we also plan to explore \textbf{general relativity effects} in orbital mechanics. Future expansions will incorporate metric-based spacetime simulations that allow players to experience phenomena such as gravitational time dilation and curved trajectories near massive bodies.

\section{Progress}

\subsection{Custom Physics Integrator}

We develop a custom physics integrator from scratch to accurately simulate multi-body gravitational interactions in our game environment. Unlike patched-conic approximations used in many simplified space simulators, our system numerically evaluates the full Newtonian $N$-body problem at every timestep:

\[
\mathbf{a}_i = \sum_{j \ne i} G m_j \frac{\mathbf{r}_j - \mathbf{r}_i}{\|\mathbf{r}_j - \mathbf{r}_i\|^3},
\]

where $\mathbf{a}_i$ is the acceleration of body $i$, and the summation accounts for the gravitational influence of all other bodies.  
To update positions and velocities, we use a semi-implicit (symplectic) Euler integrator, which significantly improves energy stability over standard explicit Euler while remaining lightweight enough for real-time rendering in JavaScript.

The integrator supports:
\begin{itemize}
    \item real-time spacecraft thrust as instantaneous or continuous $\Delta v$ inputs,
    \item trajectory prediction for forward simulation,
    \item multiple reference frames (Sun-centric, Earth-centric, and Moon-centric),
    \item stable long-duration orbits under multi-body perturbations.
\end{itemize}

This integrator forms the core of all in-game orbital mechanics and enables realistic behaviors, such as Hohmann transfers, lunar gravity assists, and perturbed multi-body trajectories.


\subsection{Graphics and UI}
We complete a full-scale modeling of the solar system and apply distinct surface textures to different celestial bodies (Scene layout, basic). We provide ambient lighting and use a point light to simulate sunlight to illuminate the planets (Environment lighting, 1pt). We plan to further refine the environmental lighting, such as incorporating skybox and HDRI-based lighting to increase realism.

\begin{figure}[H]
  \centering
  \includegraphics[width=0.9\linewidth]{pics/image4.png}
  \caption{Solar system overview, showing major celestial bodies and reference frame transformations.}
  \label{fig:solar_system}
\end{figure}

\begin{figure}[H]
  \centering
  \includegraphics[width=0.7\linewidth]{pics/image3.png}
  \caption{Earth, with past trajectories of celestial bodies and the predicted spacecraft trajectory. Earth is illuminated by sunlight and ambient lighting.}
  \label{fig:earth_trajectory}
\end{figure}

We design a clean, tech-inspired UI that presents comprehensive trajectory information and renders real-time predicted orbital paths for all celestial bodies and the spacecraft. We also provide basic user guidance (Proper game start and end interfaces, basic; User-friendly layout, 1pt). We will continue refining the UI.

\begin{figure}[!h]
  \centering
  \includegraphics[width=0.9\linewidth]{pics/ui.png}
  \caption{UI demonstration.}
  \label{fig:solar_system}
\end{figure}

\subsection{Animation}


Using the custom physics integrator, we simulate the motion of all celestial bodies and the spacecraft according to physical laws, resulting in realistic planetary motion and orbital trajectories (Kinematics, 3 pts). We plan to implement planetary rotation, spacecraft thruster effects (Simple object animation, 1pt), and collision detection with appropriate crash responses (Collision handling, 1 pt).


\begin{figure}[!h]
  \centering
  \includegraphics[width=0.7\linewidth]{pics/image5.png}
  \caption{Moon-centric reference frame visualization, with the spacecraft trajectory relative to the Moon and nearby bodies.}
  \label{fig:moon_frame}
\end{figure}

\subsection{Interactive Control}
We implement keyboard-and-mouse control for the spacecraft (Control of the main game character, basic). Players can switch between multiple cameras to observe the solar system from various perspectives or inspect the spacecraft up close (Camera motion control, basic). We plan to add camera shake effects during collisions or rapid maneuvers (2 pts).

\subsection{Gameplay Pipeline}
We implement a playable game with basic user instructions (completion, basic).
We plan to introduce increasing difficulty levels throughout gameplay (Advanced Game Features, 2 pts), save/load functionality (Advanced Game Features, 2 pts), and gamepad control (Advanced Game Features, 2 pts). We also plan to introduce synchronized engine-thrust sound effects to enhance the gameplay experience (Synchronized audio, 1pt). We will host our game on Github page for online access (completion, 2 pts).

% \subsection{Graphics Assets}
% \begin{itemize}
%     \item \textbf{Environment lighting (1pt):} Skybox and HDRI-based lighting.
%     \item \textbf{Synchronized audio (1pt)}: sound effects for engine thrust to enhance the gameplay experience.
% \end{itemize}

% \subsection{Animation}
% \begin{itemize}
%     \item \textbf{Kinematics (3pts):} Planetary motion and spacecraft trajectory simulation.
%     \item \textbf{Simple object animation (1pt):} Planet rotation, spacecraft thruster glow.
% \end{itemize}

% \subsection{Interactive Control}
% \begin{itemize}
%     \item Control of the main game character (basic)
%     \item Camera motion control in either third-person or first-person view (basic)
%     \item Implement suitable camera shakes during collisions or movement to enhance immersion (up to 2pts)
% \end{itemize}

% \subsection{UI Design}
% \begin{itemize}
%     \item Proper game start and end interfaces (basic)
%     \item Additional auxiliary interfaces (up to 2pts)
%     \item User-friendly layout with visually appealing design (1pt)
% \end{itemize}

% \subsection{Other Advanced Features}
% \begin{itemize}
%     \item \textbf{Increasing difficulty (2pts):} Mission progression: Earth $\rightarrow$ Moon $\rightarrow$ Mars $\rightarrow$ Halley.
%     \item \textbf{Fully functional game (1pt):} Prototype playable.
%     \item \textbf{Save and load system} (up to 2pts).
%     \item \textbf{Support for various controllers} (up to 2pts).
% \end{itemize}

% \subsection{Completion}
% \begin{itemize}
%     \item \textbf{A fully functional, playable game} (basic).
%     \item \textbf{User manual or instructions} (basic).
%     \item \textbf{Online access} (up to 2pts).
% \end{itemize}


\section{Detailed Schedule for Future Work}
\begin{table}[H]
\centering
\begin{tabularx}{\linewidth}{|c|X|}
\hline
\textbf{Week} & \textbf{Tasks} \\
\hline
12 & Improve trajectory integrator, implement Hohmann transfer mission UI, add camera smoothing. \\
\hline
13 & Implement lunar gravity assist mechanics, Mars mission, audio system. \\
\hline
14 & Add Halley’s Comet mission, improved skybox, motion blur and DOF refinement. \\
\hline
15 & Begin GR mode (metric-based spacetime), bug fixing. \\
\hline
16 & Final polish, UI refinement, demo preparation. \\
\hline
\end{tabularx}
\caption{Project schedule and planned tasks.}
\label{tab:schedule}
\end{table}

% \section{Simulation and Rendering Results}

% We currently have the following implemented visual and simulation components for our interactive game:

% \begin{figure}[H]
%   \centering
%   \includegraphics[width=0.7\linewidth]{pics/image1.png}
%   \caption{Hohmann transfer example. Demonstrates spacecraft trajectory from Earth to Moon.}
%   \label{fig:hohmann}
% \end{figure}

% \begin{figure}[H]
%   \centering
%   \includegraphics[width=0.7\linewidth]{pics/image2.png}
%   \caption{Sun rendering with emissive shader. Highlights the lighting effects in the scene.}
%   \label{fig:sun}
% \end{figure}

% \begin{figure}[H]
%   \centering
%   \includegraphics[width=0.7\linewidth]{pics/image3.png}
%   \caption{Earth rendering with past trajectories of celestial bodies and predicted future spacecraft trajectory. Shows combined effect of Sun light and ambient lighting.}
%   \label{fig:earth_trajectory}
% \end{figure}

% \begin{figure}[H]
%   \centering
%   \includegraphics[width=0.7\linewidth]{pics/image5.png}
%   \caption{Moon-centric reference frame visualization. Spacecraft trajectory relative to Moon and other bodies.}
%   \label{fig:moon_frame}
% \end{figure}

\section{Acknowledgement}

We acknowledge the use of Three.js, as well as publicly available NASA/JPL ephemeris data for reference.

\bibliographystyle{ACM-Reference-Format}
\bibliography{references}

\end{document}
